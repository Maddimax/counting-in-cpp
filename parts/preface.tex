%-------------------------------------------------------------------------------
\chapter{Preface}

%-------------------------------------------------------------------------------
\section{About This Book}
The goal of this document is to list most, if not all, features
introduced in the \cpp language since the first well-known deep update
in the language, also known as \cpp11, up to the most recent version
of the standard, which is \cpp20 by the time I am writing this.

These features are presented following a systematic format where the
pre-\cpp11 way is reminded to the reader, with a short explanation of
why it may have been problematic or inefficient, then the new way of
doing things is presented.

Some parts of the language will probably be silenced, mostly the parts
for which I don't know much. Some parts will be presented with harsh
critics, maybe even just based on my experience. Am I an authority on
the subject? Probably not. Am I experimented? I tend to think I am,
but you are the judge. Anyway, it is ok if you dismiss these critics,
just know that there are pros and cons for anything.

This book won't go into the details and subtleties of any feature,
nor into compiler-specific stuff. The reason being mostly time (I have
a limited time to dedicate to this book) and space (it is already
large enough). The reader is invited to satisfy his curiosity and
complete is knowledge by reading other material. For example, the
website \url{https://en.cppreference.com} has everything you need to
know about anything from the language.

\bigskip

While this book is about \cpp, one should remember that \cpp{} itself
is just a tool in the programmer's toolbox. If you are learning \cpp{}
to become a programmer, a good programmer, I would suggest to rethink
your plan and learn programming on a larger scale: algorithms, data
structures, project management, packaging, dependency management,
coding style, reviews, testing… There are many aspects to master in
the life of a programmer.

As a good starting point, every programmer should read Code Complete
by Steve McConnell. This book goes in detail in all aspects of
software development, over decades of projects, so if you read it you
will also gain part of the knowledge from these people who tried and
failed before you.

Clean Code by Robert C. Martin is a good second book to read, even
though I would not approve all suggestions. For example, intensive
factorization and the tendency to use object-oriented programming
everywhere are things I have learnt to avoid. Still, the book is a
reference in software development, so you should read it at least to
make yourself an opinion and to know what is going on in the business.

\bigskip

Finally, remember these two steps of learning and mastering: read to
acquire the experience of others, practice to build your own
experience.

%-------------------------------------------------------------------------------
\section{About The Author}

I'm just some guy who write stuff not to think about the
meaninglessness of existence. My intent is not to compete nor to show
of, but maybe are you wondering if I am relevant on the topics from
this book? See by yourself.

I write code as a hobby since 1994, and professionally since 2005. I
have at least 600'000 lines of \cpp behind me, just counting alive
lines of code from past projects I could find. I did not go into the
history of each project, so this does not include lines that have been
overwritten.

I also have coded on Java projects, some Pascal and Delphi ones,
Visual Basic too, BASIC a long time ago, on a Commodore 128 and later
under DOS. I did also a bit of Objective-C.

Actually, a good share of this code was crap.

Some code did end up well, though. One project I am proud of is a
mobile game written in \cpp, which was played by more than 500'000
people every day during more than three years. Aside from that, I also
took part on projects that were struggling to start and brought them
into a viable product. So I guess I made stuff that do not suck.

When I code I tend to think about long term and architecture. I try
not to take any shortcut and to answer the problem without expecting
to solve the future. I code in small boxes, many, with the intent that
they can be broken, removed, replaced, without changing everything.

Finally, I am certainly not the type to rush for the new thing. I like
tools and practices that have been well tested, so you probably won't
hear me telling you to use this new thing from \cpp{}42 because it's
new and it will show you are modern and blah blah blah.

\bigskip

That being said, I hope you will find something useful in this book :)
