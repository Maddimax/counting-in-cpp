%-------------------------------------------------------------------------------
\chapter{Preface}
\renewcommand*\thesection{\arabic{section}}

%-------------------------------------------------------------------------------
\section{About this Book}
The goal of this document is to list most, if not all, features
introduced in the \cpp{} language since the first well-known deep
update in the language, also known as \cpp11, up to the most recent
version of the standard, which is \cpp20 by the time I am writing
this.

These features are for the most part presented following a format
where the pre-\cpp11 way is reminded to the reader, with a short
explanation of why it may have been problematic or inefficient, then
the new way of doing things is presented.

Some parts of the language will probably be silenced, mostly the ones
for which I don't know much. Some will be presented with harsh
critics, maybe even just based on my experience. It is okay if you
dismiss these critics, just know that there are pros and cons for
anything.

This book won't go into the details and subtleties of any feature, nor
into compiler-specific stuff. The reason being mostly time (as far as
I can tell I have a limited time in my life) and space (the book is
already large enough). The reader is invited to satisfy his curiosity
and complete is knowledge by reading other material. For example, the
website \url{https://en.cppreference.com} has everything we need to
know about any feature of the language.

Finally, a basic knowledge of the language is preferable for the
reader to enjoy this book, as some notions will be used without being
explained.

\bigskip

While this book is about \cpp, one should remember that \cpp{} itself
is just a tool in the programmer's toolbox. If you are learning \cpp{}
to become a programmer, a good programmer, I would suggest to rethink
your plan and learn programming on a larger scale: computer
architecture, algorithms, data structures, project management,
packaging, dependency management, coding style, reviews, testing…
There are many aspects to be familiar with in the daily life of a
programmer.

As a good starting point, every programmer should read Code Complete
\cite{code-complete}. This book goes in detail in all aspects of
software development over decades of projects, so if you read it you
will also gain part of the knowledge from these people who tried and
failed before you.

Clean Code \cite{clean-code} is a good second book to read, even
though I would not approve all suggestions. For example, intensive
factorization and the tendency to use object-oriented programming
everywhere are rather things I have learnt use parsimoniously. Still,
the book is a reference in software development, so you should read it
at least to make yourself an opinion and to know what is going on in
the business.

\bigskip

Finally, remember that if reading is acquiring the experience of
others, practicing is building our own experience.

%-------------------------------------------------------------------------------
\section{About the Author}

Should one take this book's content at face? Maybe the author is just
some guy who just write useless stuff to forget about the
meaninglessness of existence? If one is wondering if I am relevant on
the topics from this book, I think this would help them to get an idea
of my experience.

\bigskip

First of all, I read a lot of code. Everyday. I read code on GitHub,
on blogs, on StackOverflow, Reddit, and other forums. I read code
written by me or others, from my personal projects, from my employer,
or from random projects I find on the Internet. All this code is
displayed either on my laptop, or in a terminal connected to a remote
server, or on my phone. Reading is undoubtedly the main part of my
programming activities.

On the productive side, I write code as a hobby since 1994, and
professionally since 2005. I have at least half a million of \cpp{}
behind me, just counting alive lines of code from past projects I
could find.

I also have coded on Java projects, a bit of HTML and JavaScript, and
many Bash scripts. Additionally, I coded some C\# programs, some
ActionScript, some Pascal and Delphi ones, Visual Basic too, BASIC a
long time ago, on a Commodore 128 and later under DOS. I did also a
bit of Objective-C.

Let's face it though, a good share of this code was crap.

\bigskip

Some code did end up well nonetheless. One project I am proud of is a
mobile game written in \cpp, which was played by more than 500'000
people every day during more than three years. Aside from that, I also
took part on projects that were struggling to start and brought them
into a viable product. So I guess I made stuff that do not suck.

When I code I tend to think about long term and architecture. I try
not to take any shortcut and to answer the problem without attempting
to solve the future. I code in small boxes, many, with the intent that
they can be broken, removed, replaced, without changing everything.

Finally, I am certainly not the type to rush for the new thing. I like
tools and practices that have been well tested, so you probably won't
hear me telling you to use this new thing from \cpp{42} because it's
new and it will show that you are modern and blah blah blah.

\bigskip

That being said, I hope you will find something useful in this book :)

%-------------------------------------------------------------------------------
\section{License}
This work of art is licensed under a Creative Commons
Attribution-ShareAlike 4.0 International License. See
Appendix~\ref{license} for the full license text.

\marginheader{This one.}%
%
The header file icon displayed in the margins, like the one on the
side of this paragraph, is based on the New Document icon from the
Tango Icon Theme \cite{tango-icon-theme}. The original icon is in the
public domain and in order to respect the intent of the source the
variation made for this book, as in the SVG file available in the
repository containing the sources of this book, is also released in
the public domain.

\renewcommand*\thesection{\arabic{chapter}.\arabic{section}}
