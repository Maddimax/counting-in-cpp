\subsection{Threads}

Before \cpp11 there was just nothing in the standard library to
execute a thread, so we had to either go native with pthreads, Windows
threads, or whatever fit our need, or else we could use an independent
library that would do the system abstraction for us, like Boost.Thread.

\bigskip

Thankfully, \cpp11 came with \code{std::thread}, which does the system
abstraction for us and allows us to write portable threaded
code\footnote{When the targeted platform supports it. I'm looking at
  you WebAssembly!}. Launching a thread is as simple as this:

\begin{lstlisting}
auto expensive_computation =
  []() -> void
  {
    // This infinite loop takes forever to complete! It slows down
    // our awesome app, better put it in a thread.
    while (true);
  };

@\emcode{std::thread t(expensive\_computation);}@
// Here we go, the thread is running.
\end{lstlisting}

With the threads come the mutexes, condition variables, and more, that
will help us write synchronization points. Here is a more complete
example with two threads accessing shared data:

\lstinputlisting{examples/thread/flood.cpp}
