\subsection{Tuple}
\label{tuple}

The type \code{std::pair} is a nice utility class available in the STL
since forever. It is a struct accepting two template parameters and
defining two fields of these types, named ``first'' and
``second''. Approximately like:

\begin{lstlisting}
template<typename T1, typename T2>
struct pair
{
  T1 first;
  T2 second;
};
\end{lstlisting}

It is for example the value type of associative containers like
\code{std::map}, and now we are stuck with map entries with fields
named ``first'' and ``second'' while something like ``key'' and
``value'' would have carried the semantics better. Meh. Well, it may
not be the most expressive but at least we are reusing code via this
hyper generic reusable type, hooray!

Sorry, I inadvertently switched the rant-mode button on\footnote{That
  being said, I actually love the fact that even for small types like
  that effort is made by \cpp{} developers to build more efficient
  implementations. Search for \emph{compressed pair} or \emph{tight
    pair} for example.}.

\bigskip

Generalizing this structure to any number of fields/types was quite
complex, as before the arrival of variadic templates
\ref{variadic-template} in \cpp11 we had to go through some type lists
and other metaprogramming dances.

Thanks to their introduction, generalizing \code{std::pair} to any
number of elements is now somewhat simpler, and it is already done in
the STL via \code{std::tuple}. Additionally, \code{std::make\_tuple()}
is a utility function that can create a tuple from its values,
deducing the actual tuple type from its arguments. Finally, the last
key function when using tuples is \code{std::get()}, which allows to
access a tuple element by its index.

\begin{lstlisting}
// Creating a tuple explicitly.
std::tuple<int, float, bool> t(24, 42.f, true);

// Creating a tuple from its values.
t = std::make_tuple(24, 42.f, true);

// Accessing the elements in a tuple.
printf("%f\n", std::get<1>(t));
std::get<2>(t) = false;
\end{lstlisting}

Using tuple in a \cpp{} program typically happen for few reasons:
\begin{itemize}
\item By laziness where a struct could have been used\footnote{Don't
  do that.},
\item To return multiple values from a function:
  \code{std::tuple<int, int> get\_dimensions()}\footnote{Don't do
    that either, just write a meaningful type.}.
\item To lure Python developers into writing \cpp{}\footnote{But… why
  should we lure them if it's not their kind of stuff?}.
\item To group types, to carry type lists, in metaprogramming context.
\end{itemize}

\begin{guideline}
If you ever end up in a situation where \code{std::tuple} seems to be
the best type for an aggregate value, please reconsider.

Naming things is hard, but having a named struct will be better to
carry the meaning to the next readers than presenting them a bunch of
data thrown in a bag.
\end{guideline}

It is thus quite difficult to find a good short example for
\code{std::tuple}. One good example could be argument binding, but
it's a quite long example that needs many extra features. Another
example is found in one constructor of \code{std::pair}, which allows
to directly pass the arguments to use to construct its fields:

\begin{lstlisting}
template<typename T, typename U>
struct pair
{
  // Simplified for the example.
  template<typename... FirstArgs, typename... SecondArgs>
  pair
  (std::tuple<FirstArgs...> first_args, std::tuple<SecondArgs...> second_args)
    : first(/* here we pass the content of first_args */),
      second(/* here we pass the content of second_args */)
  {}
\end{lstlisting}

Yet another example is a metaprogramming use case where multiple
parameter packs must be passed to a type or function:

\begin{lstlisting}
// This won't work since the compiler cannot tell where to split the
// As and Bs
template<typename... As, typename... Bs>
struct failing_multi_pack
{};

// The code below will work though.

// This one is just the base template declaration, not defined.
template<typename As, typename Bs>
struct working_multi_pack;

// And we can specialize it for tuples. Now the compiler can split As
// and Bs.
template<typename... As, typename... Bs>
struct working_multi_pack<std::tuple<As>, std::tuple<Bs>>
{};
\end{lstlisting}

Well, I can see that you are a bit disappointed. Let's have a look at
the binding stuff:

\lstinputlisting{examples/tuple/binding.cpp}


