%-------------------------------------------------------------------------------
\section{Delegated Constructors}

Before \cpp11, constructors cannot call other constructors, so if you
want to share initialization code between multiple constructors,
like in the example below:

\begin{lstlisting}
struct foo
{
  foo(bar* b, float f, int i)
    : m_bar(b),
      m_f(f),
      m_i(i)
  {}
  
  foo(bar* b, int i)
    // can't I call foo(b, 0, i) directly?
    : m_bar(b),
      m_f(0),
      m_i(i)
    {}

private:
  bar* const m_bar;
  const float m_f
  const int m_i;
};
\end{lstlisting}

Then you have to put it in some separate member function called by the
constructor, like this:

\begin{lstlisting}
struct foo
{
  foo(bar* b, float f, int i)
  {
    @\emcode{init}@(b, f, i);
  }
  
  foo(bar* b, int i)
  {
    @\emcode{init}@(b, 0, i);
  }

private:
  void @\emcode{init}@(bar* b, float f, int i)
  {
    m_bar = b;
    m_f = f;
    m_i = i;
  }

  // We cannot make any of these member const anymore.
  bar* m_bar;
  float m_f;  
  int m_i;
};
\end{lstlisting}

This approach was kind of error-prone. Stuff may happen before and
after the call to the init function, and actually nothing prevent it
to be called at any point in the life of the instance. Finally, this
is incompatible with const members.


Starting from \cpp11, a constructor can call another constructor:

\begin{lstlisting}
struct foo
{
  foo(bar* b, float f, int i)
    : m_bar(b),
      m_f(f),
      m_i(i)
  {}
  
  foo(bar* b, int i)
    : @\emcode{foo}@(b, 0, i)
  {}

private:
  bar* const m_bar;
  float m_f;
  int m_i;
};
\end{lstlisting}

This solves all problems.
