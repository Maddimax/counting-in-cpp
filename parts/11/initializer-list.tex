\subsection{Initializer list}
\label{initializer-list}

Initializing a container like \code{std::vector} with a predefined
sequence of values was quite verbose before \cpp11, as we had to
insert the values one by one:

\begin{lstlisting}
void transform
(std::vector<int>& values, const std::vector<int>& multipliers);

void up_down_transform(std::vector<int>& values)
{
  // Pfff it is not even const.
  std::vector<int> pattern(4);
  pattern[0] = 0;
  pattern[1] = 1;
  pattern[2] = 0;
  pattern[3] = -1;

  // Can't we have simple initialization, like we have for arrays?
  // const int pattern[4] = {0, 1, 0, -1};

  transform(values, pattern);
}
\end{lstlisting}

However, thanks to the the introduction of
\code{std::initialize\_list} in \cpp11, we can now initialize our
containers wich bracket enclosed values:

\begin{lstlisting}
void transform
(std::vector<int>& values, const std::vector<int>& multipliers);

void up_down_transform(std::vector<int>& values)
{
  const std::vector<int> pattern = {0, 1, 0, -1};
  transform(values, pattern);

  // This one also works:
  // transform(values, {0, 1, 0, -1});
}
\end{lstlisting}

It's just like uniform initialization \aref{uniform-initialization}!
Or is it?

\subsubsection{The Problem with \code{std::initializer\_list}}

Instances of \code{std::initializer\_list} are created by the compiler
when it encounters a list of values between brackets and if the target
to which these values are assigned is, or can be constructed from, an
\code{std::initializer\_list}.

In the example above, we can create a vector from the values because
\code{std::vector} defines a constructor taking an
\code{std::initializer\_list} as it sole argument. This constructor
then copies the values from the list into the vector.

\bigskip

I think it is important to emphasize that: the constructor
\emph{copies} the values from the list. There is no way it can move
them, let alone wrap them.

\bigskip

So we have something that looks surreptitiously like the good old
simple and efficient aggregate initialization, that is consequently
identical in its syntax to uniform initialization
\aref{uniform-initialization}, and that is actually a quite
inefficient way to initialize something as soon as the element type is
non trivial.

The worse part is that bracket initialization is pushed by the
so-called ``modern'' \cpp{} trend, propagating this inefficiency
everywhere.

\begin{guideline}
  Avoid \code{std::initializer\_list}.

  If you really, really, want the target container to be const, then
  use an immediately invoked lambda to construct it:

  \begin{lstlisting}
    const std::vector<int> pattern =
      []() -> std::vector<int>
      {
        std::vector<int> result(4);
        result[0] = 0;
        result[1] = 1;
        result[2] = 0;
        result[3] = -1;

        return result;
      }(); // Watch out for the parentheses here,
           // it is a call to the lambda.
  \end{lstlisting}
\end{guideline}

