\subsection{Override and Final}

So we have a class in our \cpp-98 code base that looks like that:

\begin{lstlisting}
struct abstract_concrete_base
{
  virtual ~abstract_concrete_base();
  virtual void consume();
};
\end{lstlisting}

Down the hierarchy we have some class that overrides the
\code{consume()} function\footnote{You may also find versions of this
  pattern in the wild where the \code{virtual} keyword is repeated in
  the derived classes, so the reader can tell that it is an overridden
  function without looking at the declaration of the parent class.}:

\begin{lstlisting}
struct serious_business : abstract_concrete_base
{
  void consume();
};
\end{lstlisting}

Suddenly, the base class has some work to do in \code{consume()}
regardless of the details of the derived classes. Also, it is well
known that one should prefer non-virtual public functions calling
virtual private functions, so let's update our code:

\begin{lstlisting}
struct abstract_concrete_base
{
  virtual ~abstract_concrete_base();

  void consume()
  {
    stuff_from_base_class();
    do_consume();
  }

private:
   void stuff_from_base_class();
   virtual void do_consume();
};
\end{lstlisting}

Good. Everything compiles well but all derived classes are broken. It
turns out that since \code{do\_consume()} is not a pure virtual
function, there is no problem to instantiate the class. All derived
will use the default \code{do\_consume()} and their old
\code{consume()} is now just another function of theirs.

Wouldn't it be nice if there was a way to detect the problem?

\medskip

That's why the \code{override} keyword has been introduced in
\cpp11. By modifying the initial implementation of
\code{serious\_business} as follows:

\begin{lstlisting}
struct serious_business : abstract_concrete_base
{
  void consume() @\emcode{override}@;
};
\end{lstlisting}

The programmer explicitly tells the compiler that this function is
expected to override the same function from a parent class. So when
\code{abstract\_concrete\_base} is updated and its \code{consume()}
function becomes non-virtual, the compiler will immediately complain
that the function from \code{serious\_business} does not override
anything. Good!

\bigskip

Another keyword related to inheritance and function overriding has
been introduced in \cpp11: the \code{final} keyword. When used in
place of \code{override}, it tells the compiler that the function
should not be overridden in the derived classes. Previously a virtual
function could be overridden anywhere from the top to the bottom of
the hierarchy. Now if one declaration is marked as final, the derived
classes are not allowed to redefine the function anymore.

Additionally, one can place the \code{final} keyword after the name of
a class to indicate that this class cannot be derived at all:

\begin{lstlisting}
struct serious_business @\emcode{final}@ : abstract_concrete_base
{
  void consume() override;
};
\end{lstlisting}
