\section{Iterator successors and predecessors}

There are many kind of iterators: forward iterators, that can be
incremented one step at a time with \code{operator++}, bidirectional
iterators, that can additionally be decremented with
\code{operator--}, and random access iterators, that can be
incremented or decremented by any amount at once.

When writing a function taking an iterator whose type is templated,
incrementing an iterator by more than one unit cannot be done
directly, as it would fail for non-random iterators. Prior to \cpp11,
this was done with \code{std::advance()}.

\begin{lstlisting}
template<typename Iterator, typename F>
void every_n_items
(Iterator it, std::size_t count, std::size_t step, F f)
{
  for (; count > step; count -= step)
  {
    f(*it);
    std::advance(it, step);
  }
}
\end{lstlisting}

\code{std::advance()} accepts a negative distance, in which case it
will advance… hum… backwards.

\cpp11 introduces the replacement functions \code{std::prev()} and
\code{std::next()}, which are mostly here to make things clear. The
former is to be used to move the iterator backwards, while the latter
is used to move it forward. If no distance is passed, then it defaults
to one.

Note that they return a copy of the new iterator, instead of modifying
the one received as a parameter.

\begin{lstlisting}
template<typename Iterator, typename F>
void every_n_items
(Iterator it, std::size_t count, std::size_t step, F f)
{
  for (; count > step; count -= step)
  {
    f(*it);
    it = @\emcode{std::next}@(it, step);
  }
}
\end{lstlisting}
