\subsection{Reference Wrapper}

Be it \cpp11 or before, template argument deduction never pick a
reference. So, for example, if we wanted to create an
\code{std::pair<int, int\&>}, we could not use \code{std::make\_pair}:

\begin{lstlisting}
int a;
int b;

// Ok in C++11, not before. The members of the pair are references to
// a and b.
std::pair<int, int&> pair_1(a, b);

// Not ok, std::make_pair returns an std::pair<int, int>.
std::pair<int, int&> pair_2 = std::make_pair(a, b);

// Not ok in C++11 and before, for different reasons.
std::pair<int, int&> pair_3 = std::make_pair<int, int&>(a, b);
\end{lstlisting}

The \code{std::ref()} function (as well as \code{std::cref()})
introduced in \cpp11 will help us with this problem. They both create
a reference wrapper for their argument (non const or const,
respectively), that is implicitly convertible to a raw reference.

\begin{lstlisting}
int a;
int b;

// Ok, std::make_pair returns an
// std::pair<int, std::reference_wrapper<int>>, which is itself
// convertible to std::pair<int, int&>.
std::pair<int, int&> pair_make = std::make_pair(a, @\emcode{std::ref(b)}@);
\end{lstlisting}

This is especially useful when binding variables that we don't want
to copy.

\lstinputlisting{examples/reference-wrapper/capacity_tracker.cpp}
