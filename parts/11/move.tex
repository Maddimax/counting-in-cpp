%-------------------------------------------------------------------------------
\section{Move semantics}

Copies! Copies everywhere! And dynamic allocations too!

So was the world before \cpp11. In these old days, if you had to pass
a large object to a function that needed its own instance of it, then
you would certainly had created a copy of it.

\begin{lstlisting}
struct foo
{
  foo(const std::vector<bar>& bars)
    // Here we have one allocaction,
    // plus copies of the vector elements.
    : m_bars(bars)
  {}

private:
  std::vector<bar> m_bars;
};

void baz(int n)
{
  // One allocation of n bars, plus their initialization.
  std::vector<bar> bars(n);
  
  // ...
  
  foo f(bars);
  // I don't need the bars anymore, but they are still here.
}
\end{lstlisting}

\Cpp11 introduces the move semantics and the std::move() function. In
practice it means far fewer copies, as the instances data are
transfered from one place to another.

\begin{lstlisting}
struct foo
{
  // Note the pass-by-value
  foo(std::vector<bar> bars)
    // Explicit transfer into m_bars, no allocation.
    : m_bars(@\emcode{std::move}@(bars))
  {}

private:
  std::vector<bar> m_bars;
};

void baz(int n)
{
  // The single allocation in this program.
  std::vector<bar> bars(n);

  // ...

  // I don't need the bars anymore, so I transfer them.
  foo f(@\emcode{std::move}@(bars));
}
\end{lstlisting}

%-------------------------------------------------------------------------------
\subsection{The Move Constructor and Assignment}

In order to have this working, a new type of reference was added to
the language, and classes can use this reference in constructors and
assignment operators to implement the actual transfer of data from one
instance to another.

\begin{lstlisting}
struct foo
{
  foo(foo@\emcode{\&\&}@ that)
    : m_resource(that.m_resource)
  {
    that.m_resource = nullptr;
  }

  foo& operator=(foo@\emcode{\&\&}@ that)
  {
    std::swap(m_resource, that.m_resource);
  }
};

// All of these call the constructor defined above.
foo f(foo());
foo f(construct_a_foo());
foo f(std::move(g));

// All of these call the assignment operator defined above.
f = foo();
f = construct_a_foo();
f = std::move(g);
\end{lstlisting}

The parameter of such functions is considered a temporary on the call
site. They are actually not restricted to constructors or assignment
operators and can be used in any function.

The semantics of such arguments must be interpreted as ``I {\em may}
steal your data''. The {\em may} is important here, as there is no
guarantee on the call site that the instance going through an
std::move is actually moved.

%-------------------------------------------------------------------------------
\subsection{std::move, The Subtleties}

Naming quality: \faStar\faStarO\faStarO\faStarO\faStarO. Would have
put zero stars if I could.

As a great example of ``naming things is hard'', std::move does not
move anything. It actually just marks the value as transferable, by
casting it to an rvalue-reference. See this actual implementation
exctracted from GCC 9:

\begin{lstlisting}
template<typename _Tp>
constexpr typename std::remove_reference<_Tp>::type&&
move(_Tp&& __t) noexcept
{
  return static_cast
  <
    typename std::remove_reference<_Tp>::type&&
  >(__t);
}
\end{lstlisting}

As a rule of thumb of its usage, consider two use cases:

\begin{enumerate}
\item If callee \underline{always} takes ownership of the value:
  prefer arguments passed by value.
\item If callee \underline{may} take ownership of the value: use an
  rvalue reference.
\end{enumerate}

\begin{lstlisting}
// "I need my own bar."
void foo_copy(bar b) { /* ... */ }

// "I may take ownership of b...
//  ...Unless I don't."
void foo_rvalue(bar&& b) { /* ... */ }

void baz()
{
  bar b1;
  // Valid, b1 is unchanged in baz.
  foo_copy(b1);

  // Valid, you can forget about b1 now.
  foo_copy(std::move(b1));
  
  bar b2;
  // Invalid, b2 is not an rvalue
  // foo_rvalue(b2);

  // Valid, but is b2 actually moved?
  foo_rvalue(std::move(b2));
}
\end{lstlisting}
