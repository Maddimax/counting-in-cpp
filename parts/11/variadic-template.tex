\subsection{Variadic Templates}
\label{variadic-template}

Let's design a some kind of message dispatcher using features
available before \cpp11. The use case is presented by the code below.

\lstinputlisting[firstline=3]{examples/variadic-template/main-98.cpp}

The idea is to have a \code{dispatcher} whose role is to store
functions to be called with arguments provided later. It is
parameterized by the type of the arguments. If we want to accept any
kind of function, we must accept from zero to $n$ arguments.

A typical solution for that would have been to declare the type with
multiple template parameters and to default these parameters to a type
representing the state of not being used. Then the actual
implementation would have been selected by template specialization of
a super type. In order to keep this example short we are going to
limit our implementation to functions with up to two arguments.

Here is the dispatcher with the defaulted parameters.

\lstinputlisting[firstline=7]{examples/variadic-template/dispatcher-98.hpp}

Let's have a look at the dispatcher implementation with two arguments.

\lstinputlisting[firstline=7]{examples/variadic-template/dispatcher_impl_2.hpp}

Nothing special here. Let's look at the implementation with one
argument.

\lstinputlisting[firstline=7]{examples/variadic-template/dispatcher_impl_1.hpp}

This is very similar to the previous implementation. We could
factorize some parts (\code{queue()} and \code{m\_scheduled}) by
moving them in a parent class parameterized with
\code{function\_type}, but for now let's just keep it as is.

The \code{dispatch()} function cannot be factorized though, as its
signature and the \code{m\_scheduled[i](/*args*/)} line is specific to
each version.

The implementation with no arguments is equivalently similar, so I
won't include it here. I think you got the point: this is a lot of
code, redundant code, for a feature limited to only three use cases
amongst many. Wouldn't it be better if we could put all of that in a
single implementation that would handle any number of arguments?

\bigskip

The variadic template syntax introduced in \Cpp11 provides a solution
to this problem.

\lstinputlistinghl[firstline=5]{6}{examples/variadic-template/dispatcher-11.hpp}

Here we have a single implementation accepting any number of
arguments, and thus covering all use cases from the previous
implementation and more.

\bigskip

The syntax of \code{typename... Args} defines a \emph{parameter
  pack}. This is like a template parameter but with an ellipsis.

A template with a parameter pack is called a \emph{variadic
  template}. Note that it is allowed to mix a parameter pack with
non-pack template arguments, as in \code{template<typename Head,
  typename... Tail>}.

When an expression containing a parameter pack ends with an ellipsis,
it is replaced by a comma-separated repetition of the same expression
applied to each type from the pack. For example:

\begin{lstlisting}
template<typename F, typename... Args)
void invoke(F&& f, Args&&... args)
{
  // this is equivalent to
  //   f(std::forward<Args0>(args_0),
  //     std::forward<Args1>(args_1),
  //     etc);
  f(std::forward<Args>(args)...);
}
\end{lstlisting}

Or, for another example, here is a function creating an array of the
names of the types passed as template parameters:

\begin{lstlisting}
template<typename... T>
void collect_type_names()
{
  const char* names[] = {typeid(T).name()...};
  // …
}
\end{lstlisting}
