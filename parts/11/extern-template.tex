\subsection{External Templates}

External templates are one of my favorite features from \cpp11.

When we were writing template code before \cpp11, for example a
template function, then every time the code was included in a
compilation unit then the compiler would instantiate all used
symbols. Check for example the header below:

\lstinputlisting[%
  title=some\_template-98.hpp%
]{%
  examples/extern-template/some_template-98.hpp%
}

If this header was included in two .cpp files, and its function
actually called, then its compiled code would be present in the object
file of each .cpp; something we can check with \code{nm}.

Following the example, let's compile the two files below:

\lstinputlisting[%
  title=foo-98.cpp%
]{%
  examples/extern-template/foo-98.cpp%
}

\lstinputlisting[%
  title=bar-98.cpp%
]{%
  examples/extern-template/bar-98.cpp%
}

Then \code{nm} would print:

\begin{lstlisting}[language=bash]
$ nm bar-98.cpp.o foo-98.cpp.o

bar-98.cpp.o:
000000000000001b T foo(int)
@\emcode{0000000000000036 W int some\_template<int>(int)}@

foo-98.cpp.o:
000000000000001b T bar(int)
@\emcode{0000000000000036 W int some\_template<int>(int)}@
\end{lstlisting}

Not only does this consume space (54 bytes per file here) but it also
costs a lot of work for the compiler and the linker. All of this adds
up, even for medium projects. Imagine that for each template function
or class the compiler parses the code, then it compiles it, then
writes all these bytes on disk; then all these bytes are read again by
the linker, who sorts all these duplicate symbols to keep only one.

At the end of this process, literally all the work done by the
compiler has been useless. Wouldn't it have been better not to do it
in the first place? The linker then spent more time removing the crap
rather than actually linking, and the programmer was wondering if he
could get a more powerful computer. Again.

\bigskip

Thankfully the \code{extern template} from \cpp11 allows us to skip
all this useless work. First we have to tell the compiler not to
instantiate the template by adding a single line to our header:

\lstinputlistinghl[%
  title=some\_template-11.hpp%
]{14}{%
  examples/extern-template/some_template-11.hpp%
}

This line tells the compiler not to instantiate
\code{some\_template()} when \code{T = int}. Then we add a .cpp file
where we explicitly ask for the instantiation:

\lstinputlisting[%
  title=some\_template-11.cpp%
]{%
  examples/extern-template/some_template-11.cpp%
}

That's it. Let's check with \code{nm}:

\begin{lstlisting}[language=bash]
$ nm bar-11.cpp.o foo-11.cpp.o some_template.cpp

bar-11.cpp.o:
000000000000001b T bar(int)

foo-11.cpp.o:
000000000000001b T foo(int)

some_template-11.cpp.o:
0000000000000036 W int some_template<int>(int)
\end{lstlisting}

The template function is indeed compiled in a single file, and absent
from the others.

We can push even further in this case, by having the whole body of
\code{some\_template()} in the .cpp file and only its signature in the
header. It would not only avoid the parsing of the code but, more
importantly, allow to remove from the header all include directives
related to implementation details.

\begin{guideline}
If you are writing template code for which you know some or all
instantiations, then add an \code{extern template} line for them in
the header, and explicitly instantiate them in an implementation file.
This is not always feasible, but when it can be done it should be done
\end{guideline}
