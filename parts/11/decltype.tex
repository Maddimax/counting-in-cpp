\section{Type Deduction with \code{decltype}}
\label{decltype}

During the pre-\cpp11 every variable, member, argument, etc. has to be
exactly typed. For example, if you were writing a template function
receiving another function as an argument, how could you store the
result of a call of this function in a local variable?

\begin{lstlisting}
template
<
  typename InputIt,
  typename OutputIt,
  typename Predicate,
  typename Transform,
>
void transform_if
(InputIt first, InputIt last, OutputIt out,
 Predicate& predicate, Transform& transform)
{
  for (; first != last; ++first)
  {
    /* some_type */ v = *first;
    if (predicate(v))
    {
      *out = transform(v);
      ++out;
    }
  }
}
\end{lstlisting}

Getting the correct type to put in place of \code{some\_type} was not
obvious, and required a fair share of template metaprogramming. Now,
with the \code{decltype} specifier introduced in \cpp11, the
programmer can tell the compiler to use ``the type of this
expression''.


\begin{lstlisting}
template
<
  typename InputIt,
  typename OutputIt,
  typename Predicate,
  typename Transform,
>
void transform_if
(InputIt first, InputIt last, OutputIt out,
 Predicate& predicate, Transform& transform)
{
  for (; first != last; ++first)
  {
    @\emcode{decltype(*first)}@& v = *first;
    if (predicate(v))
    {
      *out = transform(v);
      ++out;
    }
  }
}
\end{lstlisting}
