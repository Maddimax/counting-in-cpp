%-------------------------------------------------------------------------------
\section{Deleted Constructors}

Delegating constructors is a nice feature, but wat about totally
removing a constructor?

Consider the class below:

\begin{lstlisting}[language=c++]
struct scoped_listener
{
  scoped_listener(dispatcher& d, callback c)
    : m_dispatcher(&d)
  {
    m_id = m_dispatcher->connect(c);
  }

  ~scoped_listener()
  {
    m_dispatcher->disconnect(m_id)
  }

private:
  dispatcher* m_dispatcher;
  int m_id;
};
\end{lstlisting}

Creating copies of scoped\_listener does not make any sense, as all
copies would share the same m\_id and will thus trigger the same call
to disconnect(int) when destructed. See for example its usage below:

\begin{lstlisting}[language=c++]
void foo()
{
  /* ... */
  scoped_listener listener(d, c1);
  scoped_listener copy(listener);
  scoped_listener other(d, c2);

  {
    // This assignment does not call disconnect(c2).
    other = listener;
    // disconnect(c1) is called here,
    // when other goes out of scope.
  }
  // disconnect(c1) is called twice here: in the
  // destruction of listener and copy.
}
\end{lstlisting}

One would typically want to forbid copies of scoped\_listener by
disabling its copy constructor.

Before \cpp11, one solution you would find here and there was to
declare the copy constructor and assignment operator as private. The
problem was that it was still available for the class and its
friends. So the programmer would then either implement an always
failing body for this constructor, emitting an error at run time, or
just don't implement the constructor, thus emitting an error at link
time.

These solutions were kind of weak, in the sense that the error, if
any, was presented quite late for the programmer, and with a not
obvious explanation.

Now, starting with \cpp11, the constructor and operators can be
explicitly deleted:

\begin{lstlisting}[language=c++,escapechar=!]
struct scoped_listener
{
  scoped_listener(const scoped_listener&)!\emcode{ = delete}!;
  scoped_listener& operator=(const scoped_listener&)!\emcode{ = delete}!;

  scoped_listener(dispatcher& d, callback c)
    : m_dispatcher(&d)
  {
    m_id = m_dispatcher->connect(c);
  }

  ~scoped_listener()
  {
    m_dispatcher->disconnect(m_id)
  }

private:
  dispatcher* m_dispatcher;
  int m_id;
};
\end{lstlisting}

Using a deleted function will trigger a clear error from the compiler
when the call is encountered.
