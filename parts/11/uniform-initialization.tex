\subsection{Uniform initialization}
\label{uniform-initialization}

There are so many ways to initialize a variable in \cpp{} that it
became a running gag, so it's natural that a new initialization syntax
was added into \cpp11.

Anyway, let's jump to the problem. The following code is a recurring
issue in \cpp{}, so much that even experienced programmers stumble
upon it once in a while.

\begin{lstlisting}
struct foo {};

struct bar
{
  bar(foo f);

  int i;
};

int main()
{
  bar foobar(foo());
  printf("%d\n", foobar.i);

  return 0;
}
\end{lstlisting}

The question is ``what is \code{foobar} in the above code?'' Most
\cpp{} programmers will say it is a \code{bar} constructed with a
default-constructed \code{foo}. Unfortunately it is actually the
declaration of a function returning \code{bar} and taking a function
returning a \code{foo} as its single argument. This problem is known
as \emph{the most vexing parse}.

A workaround for this issue is to declare a temporary of type
\code{foo} and pass it to the constructor of \code{bar}:

\begin{lstlisting}
int main()
{
  foo f;
  bar foobar(f);
  printf("%d\n", foobar.i);
}
\end{lstlisting}

This is not very convenient, and suddenly the scope is polluted by a
useless variable. Moreover it can quickly become hard to implement
when a variable number of arguments are passed to \code{foobar}
(e.g. by forwarding the arguments of a variadic macro, or a variadic
template \ref{variadic-template}).

\bigskip

Enters the uniform initialization from \cpp11. By replacing the
parentheses with brackets in the construction of the argument, the
ambiguity is lifted.

\begin{lstlisting}
int main()
{
  bar foobar(foo{});
  printf("%d\n", foobar.i);
}
\end{lstlisting}

Used like that, the brackets mean something like ``a default-created
\code{foo}''. It can also be used to zero-initialize any variable,
which is especially nice in a template context:

\begin{lstlisting}
template<typename T>
void many_tees()
{
  // If T is a class, calls the default constructor.
  // If T is a fundamental type (e.g. int), its value is
  // whatever is in memory at &t1.
  T t1;

  // Declares a function t2 with no argument and returning
  // a T.
  T t2();

  // Seems to work. Does it?
  T t3 = T();

  // If T is a class, calls the default constructor.
  // If T is a fundamental type (e.g. int), its value is
  // the zero of this type.
  T t4{};
}
\end{lstlisting}

\subsubsection{When to Use the Uniform Initialization Syntax}

Consider the code below:

\lstinputlistinghl{43}{examples/uniform-initialization/struct.cpp}

What would be the output of this program? The highlighted line
constructs a variant of \code{foo} with what looks like the aggregate
initialization syntax, or is it uniform initialization? Maybe is it a
call to a constructor? Which one?

Here is the output of this program:

\begin{lstlisting}[language=bash]
$ a.out
.a=24, .b=42
.a=42, .b=24
.a=24, .b=24
\end{lstlisting}

So the first call is without surprise an aggregate initialization.

The second one is a call to the constructor; since there is one
defined for \code{foo\_constructor} then the aggregate initialization
is disabled. Note that before \cpp11 the compiler would have reported
an error, saying that the constructor should be used. Here it calls
the constructor silently even though it looks like an aggregate
initialization.

The last one is the worst. It creates an \code{std::initializer\_list}
\footnote{See section~\ref{initializer-list} for details about that.}
with the values, then pass it to the corresponding constructor.

This is painfully ambiguous. Unfortunately, the so-called ``modern''
\cpp{} programming trend is pushing for using bracket initialization,
maintaining ambiguities everywhere. I prefer to use it
parsimoniously. In particular, uses of \code{std::initializer\_list}
should certainly be avoided as they lure the programmer into thinking
that the assignment is an efficient aggregate initialization while it
is actually copying stuff around.

\begin{guideline}
  Use bracket initialization if:
  \begin{itemize}
  \item you want to initialize an aggregate,
  \item or you want to zero-initialize something in a context where
    you don't know for sure what the type of the thing is.
  \end{itemize}

  \emph{Do not} use bracket initialization if you want to call a
  non-default constructor.
\end{guideline}

As a side note, the bracket initialization can be used without
specifying the type, for example to assign a variable or for the
return statement of a function:

\begin{lstlisting}
struct interval
{
  interval();
  interval(int a, int b);
};

interval build_interval(bool f)
{
  interval b;

  if (f)
    b = {23, 23};

  return {42, 32};
}
\end{lstlisting}

This is terrible.

\begin{guideline}
  Mind the next reader, write what you mean.

  While the compiler can effortlessly find the type of a variable, or
  the type returned by a function, a human will either have to scan
  back toward the declarations or be helped by a tool, if they have
  one.

  Constructing a value without saying the type is ambiguous for a
  human. This is a lot of effort to put on the reader to save some
  characters on the writer's side.

  Don't build complex types without explicitly tell what you want to
  build.
\end{guideline}
