\subsection{Scoped Enumerations}

An enumeration in pre-\cpp11 code is just like a C enumeration: a list
of constants of type \code{int} in the scope containing the
enumeration. For example, note how \code{appliance::fan} can be
accessed directly and assigned to an integer in the code below:

\begin{lstlisting}
enum appliance
{
  fan,
  oven
};

int main()
{
  int v = fan;
}
\end{lstlisting}

While this is nice when we actually want the values of the
enumeration to be an alias for integer constants, for example to store
bit field flags, it can quickly become a mess when the entries of
different enumerations share the same identifier in the same scope.

\begin{lstlisting}
enum appliance
{
  fan,
  oven
};

enum follower
{
  fan,
  admirer
};
\end{lstlisting}

The above code will fail to compile with a message along the lines of
``error: ‘fan’ conflicts with a previous declaration.'' Now the
typical solution to that was to add a unique prefix to the enumerated
values, but in the times of namespaces and so, is it really a solution?

\bigskip

A scoped enumeration as introduced in \cpp11 is declared by adding the
\code{class} or \code{struct} keyword between \code{enum} and the
identifier. Additionally the storage type of the values can be defined
too:

\begin{lstlisting}
enum @\emcode{class}@ appliance
{
  fan,
  oven
};

enum @\emcode{class}@ follower : @\emcode{char}@
{
  fan,
  admirer
};
\end{lstlisting}

With these declarations the values are scoped in their respective
enumerations and do not conflict with each other; so if we want to
access a value we must now prefix it with the name of the enumeration,
like in \code{follower::fan}. Moreover, they are also not implicitly
convertible to \code{int} anymore, which may or may not always be a
good thing.

The fact that we can also define the type of their values allows for
smaller memory consumption when needed, but also add the possibility
to use longer-than-\code{int} values. Nevertheless, I usually don't
specify this type unless necessary since it has to be repeated when
the enumeration is forward declared. This repetition adds coupling and
complicates any change in the type, for something that look like
implementation details to me.

