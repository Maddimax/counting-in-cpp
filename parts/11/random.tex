\subsection{Random}

Getting random numbers in \cpp{} was historically done by using
functions from the C library. First we initialized the a global
generator via a call to \code{srand()} then the values were generated
via \code{rand()}.

Unfortunately this generator was known for being a quite poor one.

\marginheader{<random>}%
%
This has been greatly improved in \cpp11. For the initialization, we
now have \code{std::random\_device} to access a hardware-based
generator\footnote{If there is one, otherwise it would be
  software-based.}, then there are many pseudo-random number
generators, \code{std::mt19937} being a quite popular one.

\begin{lstlisting}
int main()
{
  // The random device will get a random value from the system. It is
  // certainly the best random source we can have here.
  std::random_device seed;

  // This is a pseudo-random number generator, here initialized by
  // a value obtained from the random device.
  std::mt19937 generator(seed());

  // Finally this distribution will project the random values in the
  // [1, 24] range with equiprobability for each value.
  std::uniform_int_distribution<int> range(1, 10);

  for (int i = 0; i != 1000; ++i)
    printf("%d\n", range(generator));

  return 0;
}
\end{lstlisting}

Unfortunately the generators available in the standard library are
known to be both inefficient and not very good at randomness. As an
alternative, we can check for a permuted congruential generator (PCG)
\cite{pcg}, at \url{https://www.pcg-random.org/}.
