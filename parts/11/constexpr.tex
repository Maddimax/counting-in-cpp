\subsection{\code{constexpr}}

Let's say you have some complex computation, like for example counting
the number of bits set to one in an integer:

\begin{lstlisting}
int popcount(unsigned n)
{
  return (n == 0) ? 0 : ((n & 1) + popcount(n >> 1));
}
\end{lstlisting}

What happens when you want to be able to call this function both with
run-time values and compile-time constant as arguments? In the code
below we would want the size of the array to be a constant, but as it
is written its size will be computed at run-time, which makes it a non
constant-sized array, which is not standard compliant.

\begin{lstlisting}
int main(int argc, char**)
{
  int array[popcount(45)];
  printf("%d\n", popcount(argc));
}
\end{lstlisting}

A typical solution for this problem is to implement the computation
via template classes and meta-programming:

\lstinputlistinghl{31}{examples/constexpr/constexpr-98.cpp}

This implementation works but has two major problems: first it is
incredibly verbose, second it forces us to implement the same
algorithm twice, doubling the risk of bugs and errors.

\bigskip

The \code{constexpr} keyword introduced in \cpp11 allows us to use the
same implementation for both compile-time and run-time computations.

\lstinputlisting[emph=constexpr]{examples/constexpr/constexpr-11.cpp}

This keyword can be applied to a variable or a function to explicitly
tell the compiler that it can and should be computed at compile-time
when it appears in constant expressions. It is for example totally
possible to call the \code{constexpr popcount()} function as a
template argument, like in \code{popcount<popcount<42>>()}.
