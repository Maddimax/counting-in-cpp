\subsection{Smart Pointers}

Memory allocation in \cpp{} is a tough subject, dynamic allocation
being the hardest part.

Before \cpp11, the programmer had to be very careful with the life
span of dynamically allocated memory in order to, first, be sure that
it is released at some point and, second, that no access is made to it
once it has been released, not even another release.

See how many problems could occur with this short snippet:

\begin{lstlisting}
struct foo
{
  foo(int* p)
    : m_p(p)
  {}

  ~foo()
  {
    delete m_p;
  }

private:
  int* m_p;
};

void bar()
{
  foo f1(new int);
  foo f2(f1);

  int* p1(new int);
}
\end{lstlisting}

There are two problems with this code. First, \code{foo} does not
define nor disable its copy constructor, so, by default, its
\code{m\_p} pointer will be copied to the new instance. Then, when the
original instance and its copy will be destroyed, both will call
\code{delete} on the same pointer. This is what will happen with
\code{f1} and its copy \code{f2}. In the best scenario the program
would crash here.

The second problem is \code{p1}. This pointer points to a dynamically
allocated \code{int} for which no \code{delete} is written. When the
pointer will go out of scope then there will be no way to release the
allocated memory.

%-------------------------------------------------------------------------------
\subsubsection{Self-Deleting Non-Shared Pointer}

\marginheader{<memory>}%
%
\Cpp11 introduces a pointer wrapper named \code{std::unique\_ptr},
which has the merit of automatically calling \code{delete} on the
pointer upon destruction. Applied to the previous example, it would
solve one problem and force us to find a solution for the other:

\begin{lstlisting}
struct foo
{
  foo(std::unique_ptr<int> p)
    : m_p(std::move(p))
  {}

  // The default destructor does the job.

private:
  std::unique_ptr<int> m_p;
};

void bar()
{
  std::unique_ptr<int> p1(new int);

  foo f1(std::unique_ptr<int>(new int));
  foo f2(std::move(f1));
}
\end{lstlisting}

This program is undoubtedly safer than the previous one. The copy
constructor of \code{foo} is still not defined, but it is for sure
deleted since \code{std::unique\_ptr} has no copy constructor
neither. So, by default, we cannot share the resource between two
instances, which is great. The only solution here is to either
allocate a new int for \code{f2} or steal the one from {f1}. The
latter is implemented here.

Then, for the release of \code{p1}, it is automatically done when the
variable leaves the scope, so no memory is leaked.

\bigskip

One of the best features of \code{std::unique\_ptr} is the possibility
to use a custom deleter to release the pointer. This makes this smart
pointer a tool of choice when using C-like resources.

Check for example the use case of libavcodec's
\code{AVFormatContext}. The format context is obtained via a call to
\code{avformat\_open\_input(AVFormatContext**, const char*,
  AVInputFormat*, AVDictionary**)} and must be released by a call to
\code{avformat\_close\_input(AVFormatContext**)}. With the help of
\code{std::unique\_ptr} this could be done as follows:

\begin{lstlisting}
namespace detail
{
  static void close_format_context(AVFormatContext* context);
}

void foo(const char* path)
{
  AVFormatContext* raw_context_pointer(nullptr);

  const int open_result =
    avformat_open_input
      (&raw_context_pointer, path, nullptr, nullptr);

  std::unique_ptr
  <
    AVFormatContext,
    decltype(&detail::close_format_context)
  >
  context_pointer
  (raw_context_pointer, &detail::close_format_context);

  // …
}
\end{lstlisting}

With this approach, the format context will be released via a call to
\code{detail::close\_format\_context} as soon as
\code{context\_pointer} leaves the scope. Do we know if the memory
pointed by \code{raw\_context\_pointer}? We do not, and it does not
matter. What we have here is a simple robust way to attach a release
function to an acquired resource.

%-------------------------------------------------------------------------------
\subsubsection{Self-Deleting Shared Pointer}

\marginheader{<memory>}%
%
I have a hard time trying to find a use case where
\code{std::shared\_ptr} is the best solution, so let's just focus on a
good-enough solution.

This smart pointer is the answer to the problem of having a
dynamically allocated resource that may outlive its creator, and that
may also be accessed from two independent owners. In this case, the
ownership of the resource is unclear, as it is \emph{shared}, so the
idea is to keep the resource alive until all its owner release it.

For example, let's say we have a function whose role is to dispatch a
message to multiple listeners, whom will not process it right now. For
some reason the message cannot be copied, so we cannot send a copy of
it to everyone:

\begin{lstlisting}
void dispatch_message
(const std::vector<listener*>& listeners,
 const std::string& raw)
{
  message m = parse_message(raw);
  std::shared_ptr<message> p
    (std::make_shared<message>(std::move(m)));

  for(listener* l : listeners)
    listener->add_to_queue(m);
}
\end{lstlisting}

With this implementation the message outlives the scope of
\code{dispatch\_message()} and will remain in memory until all
listeners have released it.

Is it the best solution for this problem? Honestly I doubt
that. Actually, most uses of \code{std::shared\_ptr} I have seen,
including mine, looked a bit like a lack of reasoning about resource
management.

Wouldn't it have been better if the dispatching and the listeners were
scheduled in a loop and if the dispatcher would own the messages for
some iterations, or until they would be marked as processed by the
listeners? Do we really have to pay for dynamic allocations here?

\begin{guideline}
When you find yourself thinking that a shared resource with no clear
life span — a \code{std::shared\_ptr} — is a good answer to your
problem, please double check your solution, and ask for a second
opinion.
\end{guideline}

\begin{pitfall}
Many documentations declare, rightfully, that \code{std::shared\_ptr}
is thread-safe, and I have seen people using it to \emph{safely}
access the allocated resource in a concurrent way.

\bigskip

It must be said that the only thread-safe part in a
\code{std::shared\_ptr} is its reference counter.

\bigskip

Nothing is done — nothing can be done — to provide an automatic
thread-safe access to the allocated resource. Actually, even having
two threads doing respectively a copy (i.e.\ \code{std::shared\_ptr<T>
  p(sp)}) and a deletion (i.e.\ \code{sp.release()}), for the same
shared pointer \code{sp}, has no defined outcome without additional
synchronization. The only guarantee is that the increment of the
counter in the copy won't be done between the decrement and the
deletion from the release.

\end{pitfall}

%-------------------------------------------------------------------------------
\subsubsection{Shared Pointer Observer}

\marginheader{<memory>}%
%
What happens when the instance pointed by a \code{std::shared\_ptr}
owns a \code{std::shared\_ptr} pointing to the owner of the former?
Then we have a cycle and none of the instances will be released.

\begin{lstlisting}
struct foo;

struct bar
{
  std::shared_ptr<foo> m_foo;
};

struct foo
{
  std::shared_ptr<bar> m_bar;
};

void foobar()
{
  std::shared_ptr<foo> f(std::make_shared<foo>());
  std::shared_ptr<bar> b(std::make_shared<bar>());
  b->m_foo = f;
  f->m_bar = b;

  // The instances pointed by foo and foo->bar won't be deleted.
}
\end{lstlisting}

In order to break this kind of dependency loop, one pointer should be
declared as an \code{std::weak\_ptr}. Just like
\code{std::shared\_ptr}, this smart pointer is here to point to a
shared resource, except that it does not count as an owner of the
resource. Additionally, it provides a way to test if the resource is
still available via the \code{std::weak\_ptr::lock()} function, which
returns a shared pointer on the resource if it is available, or
\code{nullptr} otherwise.

\begin{lstlisting}
struct foo;

struct bar
{
  @\emcode{std::weak\_ptr}@<foo> m_foo;
};

struct foo
{
  std::shared_ptr<bar> m_bar;
};

void foobar()
{
  std::shared_ptr<foo> f(std::make_shared<foo>());
  std::shared_ptr<bar> b(std::make_shared<bar>());
  b->m_foo = f;
  f->m_bar = b;

  {
    std::shared_ptr<foo> resource(b->m_foo.lock());
    if (resource)
      printf("This message is printed.\n");
  }

  f.reset();

  std::shared_ptr<foo> resource(b->m_foo.lock());

  if (resource)
    printf("This message is not.\n");
}
\end{lstlisting}
