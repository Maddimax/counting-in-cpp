\subsection{Rvalue References}

In the function \code{build\_widget()} below, a temporary string is
created by the code \code{label + "def"}, then a reference to this
string is passed to the constructor of \code{widget}, where it is then
copied in \code{widget::m\_label}. Finally, the temporary is deleted.

\begin{lstlisting}
struct widget
{
  widget(const std::string& label)
    : m_label(label)
  {}

  std::string m_label;
};

void build_widget()
{
  std::string label("abc");
  widget w(label + "def");
}
\end{lstlisting}

Temporaries have the interesting property that they have no use but to
be consumed by an expression building another value. In other words,
in the example above, the purpose of the result of \code{label +
  "def"} is to end up in \code{widget::m\_label}. So why do we need a
copy? It would be more efficient to just pass the instance created by
the concatenation directly to the final variable.

\bigskip

To solve this problem, \cpp11 introduces the concept of rvalue
references, which are, to put it simply, references to
temporaries. These references are denoted with a double ampersand
\code{\&\&}. See the updated example, below:

\begin{lstlisting}
struct widget
{
  widget(@\emcode{std::string\&\&}@ label)
    : m_label(label)
  {}

  std::string m_label;
};

void build_widget()
{
  std::string label("abc");
  widget f(label + "def");
}
\end{lstlisting}

The signature of the constructor indicates that it accept temporaries
as arguments. However, this code is not enough to avoid copies of the
string yet, as the instruction \code{m\_label(label)} still copies the
string. The next step is to replace this instruction with
\code{m\_label(std::move(label))}, and then we will have no copy. This
is the topic of section \ref{move}.

Note that if the constructor's argument was \code{widget}, like in
\code{widget(widget\&\& that)}, then this would declare what is called
a move-constructor. This is a distinct constructor than the
copy-constructor and it is explicitly allowed to steal the internals
of its argument, as long as the instance behind the argument stays in
a valid state.

