\subsection{Rvalue References}

In the function \code{bar()} below, a temporary string is created by
the code \code{s + "def"}, then a reference to this string is passed
to the constructor of \code{foo}, where it is then copied in
\code{foo::m\_s}.

\begin{lstlisting}
struct foo
{
  foo(const std::string& s)
    : m_s(s)
  {}

  std::string m_s;
};

void bar()
{
  std::string s("abc");
  foo f(s + "def");
}
\end{lstlisting}

Temporaries have the interesting property that they have no use but to
be consumed by an expression building another value. In other words,
in the example above, the purpose of the result of \code{s + "def"} is
to end up in \code{foo::m\_s}. So why do we need a copy? It would be
more efficient to just pass the instance created by the concatenation
directly to the final variable.

To solve this problem, \cpp11 introduces the concept of rvalue
references, which are, to put it simply, references to
temporaries. These references are denoted with a double ampersand
\code{\&\&}. See the updated example, below:

\begin{lstlisting}
struct foo
{
  foo(@\emcode{std::string\&\&}@ s)
    : m_s(s)
  {}

  std::string m_s;
};

void bar()
{
  std::string s("abc");
  foo f(s + "def");
}
\end{lstlisting}

The signature of the constructor indicates that it accept temporaries
as arguments. This code is not enough to avoid copies of the string
yet, as the instruction \code{m\_s(s)} does copy the string. The next
step is to replace this instruction with \code{m\_s(std::move(s))},
and then we will have no copy. This is the topic of section
\ref{move}.

Note that if the constructor's argument was \code{foo}, like in
\code{foo(foo\&\& that)}, then this would declare what is called a
move-constructor. This is a distinct constructor than the
copy-constructor and it is explicitly allowed to steal the internals
of its argument, as long as the instance behind the argument stays in
a valid state.

