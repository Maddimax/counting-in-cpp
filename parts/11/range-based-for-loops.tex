%-------------------------------------------------------------------------------
\section{Range-based For Loops}
\label{range-based-for-loops}

Before \cpp11, if someone wanted to iterate over a container, he had
to construct an iterator, of the correct type with respect to the
container, and write the loop with the classical three steps:
initialization (get an iterator on the begining of the container),
stopping condition (the iterator is not at the end), and the loop
increment.

\begin{lstlisting}
void foo(std::vector<int>& v, int c)
{
  // I'm going to iterate over v, so I need an iterator.
  typedef std::vector<int>::iterator it_t;
  const it_t end(v.end());

  for (it_t it(v.begin()); it != end; ++it)
    *it *= c;
}
\end{lstlisting}

All of this was quite verbose and repetitive when using standard
containers. Starting from \cpp11, all this administrative stuff can be
avoided by using a ranged-based for loop.

\begin{lstlisting}
void foo(std::vector<int>& v, int c)
{
  // Just get all entries from v.
  for (int& vi : v)
    *it *= c;
}
\end{lstlisting}

This format uniformizes iteration over anything having a begining and
an end. Moreover, it handles the typical subtelties of for loops for
you: the end of the container is guaranteed to be computed only once
and the increment use the preincrement operator.
