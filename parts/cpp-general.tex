%-------------------------------------------------------------------------------
\chapter{The \Cpp{} Language and its Community}

\Cpp is a quite old programming language: it was created by Bjarne Stroustrup
in 1982. It initially thought as an improved C.

The first official specifications were \emph{The C++ Programming
  Language} \cite{the-cpp-programming-language-1st}. This book was
then updated multiple times \cite{the-cpp-programming-language-2nd},
\cite{the-cpp-programming-language-3rd},
\cite{the-cpp-programming-language-se}, and
\cite{the-cpp-programming-language-4th}.

Starting from 1998, the official specifications are described in
\emph{The Standard}.

%-------------------------------------------------------------------------------
\section{The Standard}

The standard from 1998 set the ground for a new era of compilers by
defining how \cpp{} programs should behave, and by describing the
content of the standard library as well as the constraints on its
implementation.

The standard is defined by {\em The ISO C++ Committee}, i.e. people
from the industry: Google, HP, Oracle, Intel and many more.

It is worth noting that no code is provided by the committee. The
standard just describes the language, then independent developers
(mostly compiler vendors) provide the implementation. Theoretically,
developers can switch easily from one compiler to the other. Clients
are not tied to the vendor's compiler anymore.

As I write there are six revisions for this document, informally
identified by the year they came out:

\begin{itemize}
\item {\bf \Cpp98} defined the core features: the syntax, the memory
  model, templates, namespaces… And the Standard Template Library
  (STL): \code{std::vector}, \code{std::string}, \code{std::map}…
\item {\bf \Cpp03} fixed some wording and inconsistencies.
\item {\bf \Cpp11} the beginning of what is called \emph{modern
  \cpp}. Initially expected during the 00's, the committee had to kiss
  goodbye to some awaited feature. Better done than perfect.
\item {\bf \Cpp14} mostly bug fixes but also nice features:
  e.g. variable templates.
\item {\bf \Cpp17} nice new features: fold expressions, \code{if
  constexpr}, copy elision, \code{std::optional}, \code{
  string\_view}…
\item {\bf \Cpp20} \code{std::span}, concepts, modules… Also: your
  compiler still doesn't fully support \cpp17.
\end{itemize}

%-------------------------------------------------------------------------------
\section{The Community}

The \cpp{} community is made of humans, so it is naturally composed of
a lot of good things and many problems. Sometimes simultaneously. And
since the language allows several paradigms and provides multiple
tools, there are approximately as many programming styles than there
are \cpp{} programmers.

Two typical behaviours appeared very common to me in the recent
years. The first one is about the tools provided by the standard
library. First you hear people complaining: ``The standard does not
even contain {\em feature}. One has to use {\em libfeature} for it.''
Cue to the release of the aforementioned feature, and suddenly: ``The
specs for {\em feature} in the standard prevent efficient
implementations. One has to use {\em libfeature} for it.''

The second behaviour is about build times. \Cpp{} is well known for
being the language of quite long to compile programs, especially when
the program contains templates or other metaprogramming
techniques. This is a topic that regularly comes as a major pain; it
was even listed as the second most frustrating thing in the 2021
Annual C++ Developer Survey
\cite{2021-annual-cpp-developer-survey}. Nevertheless, once you are on
the field you will meet developers being like ``Here's a header-only
library for {\em feature} relying heavily on templates and
metaprogramming stuff.'' Then you will hear complaints like
``Compilation times are too long!  The committee should do something
about that!''. Finally, there will be even other developers asking for
more header-only libraries because it is easy to integrate in a
project\footnote{Dependency management in \cpp{} is the main
  frustrating thing according to the aforementioned survey.}.

\bigskip

In one case, people's demands are ignored and they complain about
it. Then they receive what they asked for, and they still complain
about it.

The second case is more about resource management. People have a
limited computing power, and they use every drop of it until it
becomes unbearable. Then they ask others to solve their problem. They
also buy even more computing power, and still use every drop of
it. Doesn't it look suspiciously like some other real-life important
resource management problems?

\bigskip

Humans…
