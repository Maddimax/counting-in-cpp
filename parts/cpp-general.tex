%-------------------------------------------------------------------------------
\chapter{The \Cpp{} Language and its Community}

\Cpp is a quite old programming language: it was created by Bjarne
Stroustrup in 1982. It was initially thought as an improved C.

The first official specifications were \emph{The C++ Programming
  Language} \cite{the-cpp-programming-language-1st}. This book was
then updated multiple times \cite{the-cpp-programming-language-2nd},
\cite{the-cpp-programming-language-3rd},
\cite{the-cpp-programming-language-se}, and
\cite{the-cpp-programming-language-4th}.

Starting from 1998, the official specifications are described in
\emph{The Standard}.

%-------------------------------------------------------------------------------
\section{The Standard}

The standard from 1998 set the ground for a new era of compilers by
defining how \cpp{} programs should behave, and by describing the
content of the standard library as well as the constraints on its
implementation.

The standard is defined by {\em The ISO C++ Committee}, i.e. people
from the industry: Google, HP, Intel, Oracle, and many more.

It is worth noting that no code is provided by the committee. The
standard just describes the language, then independent developers
(mostly compiler vendors) provide the implementation. Theoretically,
developers can switch easily from one compiler to the other. Clients
are not tied to the vendor's compiler anymore.

As I write there are six revisions for this document, informally
identified by the year they came out:

\begin{itemize}
\item {\bf \Cpp98} defined the core features: the syntax, the memory
  model, templates, namespaces… And the Standard Template Library
  (STL): \code{std::vector}, \code{std::string}, \code{std::map}…
\item {\bf \Cpp03} fixed some wording and inconsistencies.
\item {\bf \Cpp11} the beginning of what is called \emph{modern
  \cpp}. Initially expected during the 00's, the committee had to kiss
  goodbye to some awaited features. Better done than perfect.
\item {\bf \Cpp14} mostly bug fixes but also nice features:
  e.g. variable templates.
\item {\bf \Cpp17} nice new features: fold expressions, \code{if
  constexpr}, copy elision, \code{std::optional}, \code{
  string\_view}…
\item {\bf \Cpp20} \code{std::span}, concepts, modules… Also: your
  compiler still doesn't fully support \cpp17.
\end{itemize}

%-------------------------------------------------------------------------------
\section{The Community}

The \cpp{} community is made of humans, so it is naturally composed of
a lot of good things and many problems. Sometimes simultaneously. And
since the language allows several paradigms and provides multiple
tools, there are approximately as many programming styles than there
are \cpp{} programmers.

Two typical behaviours appeared very common to me in the recent
years. The first one is about the tools provided by the standard
library. First we hear people complaining: ``The standard does not
even contain {\em feature}. One has to use {\em libfeature} for it.''
Cue to the release of the aforementioned feature, and suddenly: ``The
specs for {\em feature} in the standard prevent efficient
implementations. One has to use {\em libfeature} for it.''

From where I stand, it looks like people's demands are ignored, and
they complain about it. Then they receive what they asked for, and
they obviously still complain about it. To be fair, it is probably not
the same people who complain in each situation. So I hope. Are they?

The second behaviour is about build times. \Cpp{} is well known for
being the language of quite long to compile programs, especially when
the program contains templates or other metaprogramming
techniques. This is a topic that regularly comes out as a major pain;
it was even listed as the second most frustrating thing in the 2021
Annual C++ Developer Survey
\cite{2021-annual-cpp-developer-survey}.

Despite that, once we are on the field we meet developers being like
``Hey, here's a header-only library for {\em feature} relying heavily
on templates and metaprogramming stuff.'' Simultaneously, we hear
complaints like ``Compilation times are too long! The committee should
do something about that!''. And still, more developers continue asking
for more header-only libraries because it is easy to integrate in a
project\footnote{Dependency management in \cpp{} is the main
  frustrating thing according to the aforementioned survey.}.

This is a problem of resource management. People have a limited
computing power, and they use every drop of it to compile and
recompile the same heavily template-based stuff again and again, until
it becomes unbearable. At this point, instead of reviewing their work,
they ask others (the committee, the standard, compiler vendors) to
solve their problem. Eventually they may also even buy more computing
power… only to push until using every drop of it. Doesn't it look
suspiciously like some other real-life important resource management
problems?

Ah, humans…

\bigskip

Aside from these little problems, that are actually as much inherent
to the language as they are the effect of human behaviour, there is a
very large and diverse ecosystem surrounding \cpp. Many nice libraries
and tools are published, often in a free software way, and every where
we can see that people want to do their best. This is very
stimulating.

%-------------------------------------------------------------------------------
\section{The Cost of Modernity}

With all these nice updates coming every three years, as \cpp{}
developers we naturally wonder if we can use the most recent features
or if we should stick with an older version.

There is a trend amongst us to push toward using the most recent
features, visible in online discussions or by the amount of libraries
whose description is along the line of ``A \cpp{20} library to do
something or something else''. This trend is called \emph{``modern''
  \cpp{} programming}, where ``modern'' unfortunately changes almost
every day.

It is tempting to go for the most recent features for several reasons:
to benefit from the increasing performance and support from compilers,
to write more straightforward code, to make a difference with the old
conservative folks… And obviously because it is fun. As developers we
are subject to a common disease: the refactoring, also known as ``just
burn everything and rewrite from scratch''.

On the other hand, using the bleeding-edge features has some
downsides: their implementation did not go through as much testing
than the old methods, and they may even not be available everywhere
yet. Moreover, if we write libraries, we must take into account that
not all projects can afford to use the most recent
features\footnote{Specialized tooling, weird architectures, politics…
  All of them may be valid explanations for inertia.}. Consequently,
when we choose a recent standard, we explicitly exclude a bunch of
project from using it.

On this topic, I would like to share a short experience I had.

\bigskip

Near 2014 I had been working on mobile games developed in \cpp{} and
available on iOS and Android (and unofficially on Linux and OSX for
the developers). \cpp11 was ready since a long time when the first
project was started so we naturally used it. Using
\code{std::unordered\_map} and the likes was quite common in the
projects.

At some point the Android version of a game started to take forever to
launch. It happened that the filling of a large
\code{std::unordered\_set} was the issue, and after digging a bit I
found why it was problematic only on Android. It turned out that the
Android Native Development Kit (NDK) was using GCC 4.7, for which
there was a bug that caused poor performance of the insertions in
unordered maps and
sets\footnote{\url{https://gcc.gnu.org/bugzilla/show_bug.cgi?id=54075}}.

Actually we probably had poor performance everywhere these
containers were used, but only one pathological case made it
visible. At the time we managed to work around the issue by using the
equivalent containers from Boost but the taste was bitter. We were
stuck with a broken compiler for Android and the solution we chose was
to tighten an already questioned and large dependency.

\bigskip

Could we have solved this problem by switching to a different or more
recent compiler? Probably later, but not at the moment. There was
indeed some experimental support for Clang in the NDK even though the
official way to go native was via GCC, so we did try that, and it
required a lot of work to compile the small shared common part of the
project. On the other hand, we were already using Boost so switching
to their container was simple, plus it guaranteed equivalent
performance on every platform.

Regarding the standard in use, in 2014 \cpp14 was barely ready and
certainly not available in the NDK. Moreover, to put things in
perspective, consider that even at the dawn of 2021, \cpp17 was still
not fully supported by the NDK\footnote{The revision r21e released in
  January lacks support for \code{std::filesystem}. This has been
  added in revision r22b, released in March the same year. Note that
  the former is a long term support (LTS) release, which means that it
  is expected to not have full \cpp17 for a long time.}. So, for this
platform at least, which I have heard is quite popular, pushing for
the trending features is counter productive as it can actually prevent
our tools to be used.

So should we stay with the oldest tools for maximum support? Look fook
example at Curl or zlib to name a few, they are here since forever and
work perfectly, should we come back to the \emph{good old} C? Well,
probably not, but I guess there is a compromise to make between
cutting-edge and widely accepted.

\begin{guideline}
  There is an old saying going by ``use the right tool for the
  job''. Even though it is nowadays used by programmers to dismiss the choices
  other have made, there is some wisdom in it.

  When facing the question of whether or not to use a feature from a
  somewhat recent version of the language, take the time to consider
  what it deprives you and your users it terms of usability; then
  weight the benefit you gain from its usage.

  Know that apart from some details, code written for all older
  versions of \cpp{} are still valid in newer versions. On the
  contrary, the more recent are the features you use, the less your
  code can be imported into other projects.

  So, if you chose \cpp14 just to be able to write \code{auto foo() ->
    int \{\}} instead of \code{int foo() \{\}}, or if you chose \cpp20
  just to be able to use a pair of concepts, please reconsider, as
  these features have no benefit for the final product.
\end{guideline}
